%----------------------------------------------------------------------------------------
%   PACKAGES AND THEMES
%----------------------------------------------------------------------------------------

\documentclass{beamer}

\mode<presentation> {
\usetheme{Madrid}
\usecolortheme{crane}
}

\usepackage{graphicx} 
\usepackage{booktabs} 
\usepackage[italian]{babel}
\usepackage[utf8]{inputenc}
\usepackage[T1]{fontenc}

%----------------------------------------------------------------------------------------
%   TITLE PAGE
%----------------------------------------------------------------------------------------

\title[Lambda in Java8]{Lambda Expressions in Java8: Uso e Tipaggio} 
\author[]{Mattia D'Autilia - 5765968 - mattia.dautilia@stud.unifi.it,\\ Alex Foglia - 6336805 - alex.foglia@stud.unifi.it}
\date{05/05/2018} 

\begin{document}

\begin{frame}
	\begin{center}
  		\includegraphics[width=0.2\textwidth]{assets/logo-unifi.png}
  	\end{center}
	\titlepage 
\end{frame}

\begin{frame}
	\frametitle{Indice presentazione}
	\begin{enumerate}
		\item
			\textbf{Introduzione Java e Java8}\\\
		\item
			\textbf{Lambda Calcolo vs Lambda Java}\\\
		\item
			\textbf{Tipaggio Lambda Java}\\\
	\end{enumerate}
\end{frame}

%------------------------------------------------

\begin{frame}
	\frametitle{Java}
	\begin{itemize}
		\item
			\textbf{Java} è un linguaggio di programmazione di alto livello, principalmente \textit{orientato agli oggetti}, ma accetta anche altri paradigmi come quello \textit{funzionale} ed è a \textit{tipizzazione statica}.\\\
		\item
			E' stato creato per soddisfare cinque obiettivi primari:
			\begin{enumerate}
				\item
					Essere "semplice e familiare";
				\item
					Essere "robusto e sicuro";
				\item
					Essere indipendente dalla piattaforma, da qui il detto \textit{"Write one, run everywhere"};
				\item
					Contenente strumenti e librerie per il networking;
				\item
					Essere progettato per eseguire codice da sorgenti remote in modo sicuro.
			\end{enumerate}
	\end{itemize}
\end{frame}

%------------------------------------------------

\begin{frame}
\end{frame}

%------------------------------------------------

\begin{frame}
\end{frame}

%------------------------------------------------
\section{Second Section}
%------------------------------------------------

\begin{frame}
\end{frame}

%------------------------------------------------

\begin{frame}
\end{frame}

%------------------------------------------------

\begin{frame}
\end{frame}

%------------------------------------------------

\begin{frame}
\end{frame}

%------------------------------------------------

\begin{frame}
\end{frame}

%------------------------------------------------

\begin{frame}
\end{frame}

%------------------------------------------------

\begin{frame}
\end{frame}

%----------------------------------------------------------------------------------------

\end{document}
              