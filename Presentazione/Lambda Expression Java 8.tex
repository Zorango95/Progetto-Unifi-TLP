%----------------------------------------------------------------------------------------
%   PACKAGES AND THEMES
%----------------------------------------------------------------------------------------

\documentclass{beamer}

\mode<presentation> {
\usetheme{Madrid}
\usecolortheme{crane}
}

\usepackage{graphicx} 
\usepackage{booktabs} 
\usepackage[italian]{babel}
\usepackage[utf8]{inputenc}
\usepackage[T1]{fontenc}
\usepackage{setspace}

%----------------------------------------------------------------------------------------
%   TITLE PAGE
%----------------------------------------------------------------------------------------

\title[Lambda in Java 8]{Lambda Expressions in Java 8: Uso e Tipaggio} 
\author[]{Mattia D'Autilia - 5765968 - mattia.dautilia@stud.unifi.it,\\ Alex Foglia - 6336805 - alex.foglia@stud.unifi.it}
\date{}

\begin{document}

%------------------------------------------------

\begin{frame}
	\frametitle{\textbf{Presentazione Seminario ATPL : 2017/2018}}
	\begin{center}
  		\includegraphics[width=0.2\textwidth]{assets/logo-unifi.png}
  	\end{center}
	\titlepage 
\end{frame}

%------------------------------------------------

\begin{frame}
	\frametitle{\textbf{Indice presentazione}}
	\begin{enumerate}
	\begin{LARGE}
		\item
			\textbf{Java e Java 8}\\\
		\item
			\textbf{Lambda Expression Java 8}\\\
	\end{LARGE}
	\end{enumerate}
\end{frame}

%------------------------------------------------

\begin{frame}
	\frametitle{\textbf{1 : Java e Java 8}}
	\begin{center}
		\textbf{\Huge Java e Java 8}
	\end{center}
\end{frame}

%------------------------------------------------

\begin{frame}
	\frametitle{\textbf{1.1 : Java}}
	\begin{itemize}
		\item
			\textbf{Java} è un linguaggio di programmazione di alto livello, principalmente \textit{orientato agli oggetti}, ma accetta anche altri paradigmi come quello \textit{funzionale} ed è a \textit{tipizzazione statica}.\\\
		\item
			E' stato creato per soddisfare cinque obiettivi primari:
			\begin{enumerate}
				\item
					Essere "semplice e familiare";
				\item
					Essere "robusto e sicuro";
				\item
					Essere indipendente dalla piattaforma, da qui il detto \textit{"Write one, run everywhere"};
				\item
					Contenente strumenti e librerie per il networking;
				\item
					Essere progettato per eseguire codice da sorgenti remote in modo sicuro.
			\end{enumerate}
	\end{itemize}
\end{frame}

%------------------------------------------------

\begin{frame}
	\frametitle{\textbf{1.2 : Evoluzione di Java}}
		\begin{itemize}
			\item
				Quando Java nacque nel 1995, era un linguaggio molto semplice. Con il passare degli anni sono state introdotte gradualmente tante caratteristiche, diventando un linguaggio sempre più potente e completo,in particolare con la versione 5 e 7. Quello che però non era mai cambiato sino ad ora, era la coerenza d'essere un linguaggio orientato agli oggetti.
			\item 
				Negli ultimi anni però la scena della programmazione mondiale è cambiata. In particolare con l'avvento di processore multi-core nell'uso domestico, la \textit{programmazione funzionale} è stata rivalutata. Con linguaggi moderni come \textit{Scala e Groovy} è possibile scrivere algoritmi con un numero di righe nettamente inferiore, rispetto a quello che si poteva fare con Java, che qualcuno stava già definendo un linguaggio morto in quanto con le versioni 6 e 7 aveva solo modernizzato alcune librerie, estromettendo le tanto richieste \textit{Lambda Expression}.
		\end{itemize}
\end{frame}

%------------------------------------------------

\begin{frame}
	\frametitle{\textbf{1.3 : Java 8}}
	\begin{itemize}
		\item
			Con l'avvento di \textbf{Java 8} fu però apportata una vera e propria rivoluzione, la più innovativa in tutta la storia di Java. Con l'introduzione delle \textit{Espressioni Lambda} e la possibilità di \textit{referenziare i metodi}, la \textbf{filosofia funzionale}, fa il suo ingresso nella programmazione Java.
		\item
			Ora vedremo come affrontare la nuova sfida, che è quella di far convivere i due paradigmi, quello \textit{orientato agli oggetti} e quello \textit{funzionale}, in modo tale da ottenere il meglio della programmazione.				
	\end{itemize}
\end{frame}

%------------------------------------------------

\begin{frame}
	\frametitle{\textbf{2 : Lambda Expression Java 8}}
	\begin{center}
		\textbf{\Huge Lambda Expression Java 8}
	\end{center}
\end{frame}

%------------------------------------------------

\begin{frame}
	\frametitle{\textbf{2.1 : Definizione}}
	\begin{itemize}
			\item
				Una \textbf{Lambda Expression} è detta:\\\
				\begin{itemize}
					\item
						\textbf{funzione anonima} (in inglese \textbf{"anonymous function"}), in quanto si tratta proprio di una funzione, quindi non è un metodo appartenente a una classe e chiamato tramite un oggetto, ma è una funzione senza nome;
					\item
						\textbf{clouser} (in italiano \textbf{"chiusure"}), in quanto fa uso di variabili che non sono parametri e non sono variabili locali al blocco di codice che definisce l'espressione.\\\
				\end{itemize}
			\item
				Inoltre le \textbf{Lambda Expression} permettono:\\\
				\begin{itemize}
					\item
						di scrivere codice più semplice, leggibile e meno verboso;
					\item
						di adottare nuovi pattern di programmazione, basati sulle funzioni di \textbf{ordine superiore}.
				\end{itemize}				
	\end{itemize}
\end{frame}

%------------------------------------------------

\begin{frame}
	\frametitle{\textbf{2.2 : Sintassi}}
	\begin{itemize}
		\item
			In Java 8 la sintassi generale di una funzione è la seguente:\\\
			\begin{center}
				\Large ([lista di parametri])$\rightarrow$\{blocco di codice\}\\\
			\end{center}
		\item 
			Il vantaggio principale nell'uso di una \textit{Lambda Expression}, risiede nella sinteticità dell'espressione, in quanto è possibile \textbf{omettere} :\\\
			\begin{enumerate}
				\item
					le parentesi graffe del blocco di codice quando questo è costituito da un'unica riga, ed è anche possibile evitare il ";" alla fine dello statement;\\
				\item
					il \textbf{tipo dei parametri} quando non c'è possibile di errore;\\
				\item
					le parentesi tone che circondano la lista dei parametri, nel caso quest'ultima fosse costituita da un unico elemento;\\
				\item
					nel caso il blocco di codice prevede solo un'istruzione di tipo \textit{return}, allora la parola \textit{return} è superflua non si deve proprio usare (pena errore in compilazione).					
			\end{enumerate}
	\end{itemize}	
\end{frame}

%------------------------------------------------

\begin{frame}
	\frametitle{\textbf{2.2 : Sintassi}}
	\begin{itemize}
		\item
			Abbiamo visto come una funzione viene definita in \textbf{Lambda Calcolo}. Facciamo l'esempio più semplice, la funzione identità:
			\begin{center}
				\Large$\lambda x.x$
			\end{center}
			\begin{itemize}
				\item
					\Large$\lambda$: rappresenta l'\textit{astrazione};
				\item
					la prima \Large $x$ : rappresenta la \textit{variabile di input};
				\item
					la seconda \Large $x$ : rappresenta il  \textit{corpo della funzione}.\\\
			\end{itemize}	
		\item
			Prendendo in considerazione le regole prima elencate per la sintassi di una \textit{Lambda Expression} in Java 8, possiamo vedere come tale \textit{funzione} viene scritta:
			\begin{center}
				\Large$x\rightarrow(x);$
			\end{center}									
	\end{itemize}
\end{frame}

%------------------------------------------------

\begin{frame}
	\frametitle{\textbf{2.3 : Quando usare le Lambda Expression}}
	\begin{itemize}
		\item
			Dovremmo usare le \textit{Lambda Expression,} quando il nostro obiettivo è quello di passare in maniera dinamica un certo algoritmo ad un altro metodo. Questo serve per eseguire l'algoritmo in un contesto definito dal metodo a cui stiamo passando l'algoritmo. \\\
		\item
			In generale, passare una \textit{Lambda Expression}, significherà far decidere al metodo a cui abbiamo passato il codice, il momento in cui eseguirlo.	 
	\end{itemize}
\end{frame}

%------------------------------------------------

\begin{frame}
	\frametitle{\textbf{2.4 : Funzioni di ordine superiore}}
	\begin{itemize}
		\item
			Nello studio del \textit{Lambda Calcolo}, abbiamo visto come le funzioni sono entità di prima classe: possono essere argomenti o anche risultato di una funzione.\\\
		\item
			Le \textbf{funzioni di ordine superiore} (in inglese \textbf{"higher order functions"}), sono funzioni che ammettono funzioni come argomento o risultato.
	\end{itemize}
\end{frame}

%------------------------------------------------

\begin{frame}
	\frametitle{\textbf{2.4 : Funzioni di ordine superiore}}
	\begin{itemize}
		\item
			In \textit{Java 8} posso usare una \textit{Lambda Expression} per:\\\
			\begin{itemize}
				\item
					\textbf{Assegnarla a una referenza} : trattarla come valore;\\\
				\item
					\textbf{Passarla come parametro} : parametrizzarla come comportamento;\\\
				\item
					\textbf{Ottenerla come risultato di una valutazione} : come un qualunque oggetto.
			\end{itemize}
	\end{itemize}
\end{frame}

%------------------------------------------------

\begin{frame}
\end{frame}

%----------------------------------------------------------------------------------------

\end{document}
              